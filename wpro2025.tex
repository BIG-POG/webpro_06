\documentclass[uplatex]{jsarticle}
\usepackage{amsmath}
\usepackage[dvipdfmx]{graphicx}

\usepackage[uplatex,deluxe]{otf} % UTF
\usepackage[noalphabet]{pxchfon} % must be after otf package
\setcounter{tocdepth}{3}
\usepackage{float}
\usepackage{moreverb}
\usepackage{lscape}
%\pagestyle{empty}
%\usepackage{wrapfig}

%\usepackage{EasyLayout}
\usepackage{listings}
\usepackage{ascmac}

% --- 追加設定 ---
\usepackage{geometry}
\usepackage{xcolor}
\usepackage[hidelinks]{hyperref}
\usepackage{verbatimbox}
\usepackage{color}

% --- ページレイアウト ---
\geometry{a4paper, margin=1in}

% --- listings (コードブロック) の設定 ---
\lstset{
    language=C,
    basicstyle=\small\ttfamily,
    keywordstyle=\color{blue},
    commentstyle=\color{green!40!black},
    stringstyle=\color{purple},
    showstringspaces=false,
    breaklines=true,
    frame=tb,
    backgroundcolor=\color{black!5},
    captionpos=b,
    numberstyle=\tiny\color{gray},
    numbers=left,
    stepnumber=1
}

% --- ドキュメント開始 ---
\begin{document}
\title{webプログラミング仕様書}
\author{25G1088 津村 大樹}
\date{\today}

\maketitle
\section{Github上のリポジトリのURL}
\url{https://github.com/BIG-POG/webpro_06}
\section{開発者向け仕様書}
\subsection{にじさんじSEEd'sメンバー一覧}
\subsubsection{概要}
本システムではVTuder事務所であるにじさんじの元SEEd's一期生の情報を管理し,
REST APIの原則に従って,データの一覧,詳細表示,追加,変更,削除の5つの操作を提供するシステムである.
\subsubsection{データ構造}
データ構造は以下の表\ref{1-1}のようになる.
\begin{table}[H]
\centering
\begin{tabular}{|c|c|c|}
\hline
項目 & データ型 & 説明 \\
\hline
id & 整数 & メンバーのID \\
name & 文字列 & メンバーの名前 \\
year & 文字列 & メンバーの活動期間 \\
species & 文字列 & メンバーの種族 \\
birthday & 文字列 & メンバーの誕生日 \\
fanmark & 文字列 & メンバーのファンマーク \\
fanname & 文字列 & メンバーのファンネーム \\
\hline
\end{tabular}
\caption{にじさんじSEEd'sメンバーのデータ構造}
\label{1-1}
\end{table}
表\ref{1-1}に示すデータはnijisanji\_seedsに保存し,メンバーのID,名前,活動期間,種族,誕生日,ファンマーク,ファンネームを管理する.

\subsubsection{ページ遷移}

\subsection{ARMORED CORE6登場AC一覧}

\end{document}