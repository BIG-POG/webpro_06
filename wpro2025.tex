\documentclass[uplatex]{jsarticle}
\usepackage{amsmath}
\usepackage{graphicx}
\usepackage[dvipdfmx]{color}

\usepackage[uplatex,deluxe]{otf} % UTF
\usepackage[noalphabet]{pxchfon} % must be after otf package
\setcounter{tocdepth}{3}
\usepackage{float}
\usepackage{moreverb}
\usepackage{lscape}
%\pagestyle{empty}
%\usepackage{wrapfig}

%\usepackage{EasyLayout}
\usepackage{listings}
\usepackage{ascmac}

% --- 追加設定 ---
\usepackage{geometry}
\usepackage{xcolor}
\usepackage[hidelinks]{hyperref}
\usepackage{verbatimbox}
\usepackage{color}

% --- ページレイアウト ---
\geometry{a4paper, margin=1in}

% --- listings (コードブロック) の設定 ---
\lstset{
    language=C,
    basicstyle=\small\ttfamily,
    keywordstyle=\color{blue},
    commentstyle=\color{green!40!black},
    stringstyle=\color{purple},
    showstringspaces=false,
    breaklines=true,
    frame=tb,
    backgroundcolor=\color{black!5},
    captionpos=b,
    numberstyle=\tiny\color{gray},
    numbers=left,
    stepnumber=1
}

% --- ドキュメント開始 ---
\begin{document}
\title{webプログラミング仕様書}
\author{25G1088 津村 大樹}
\date{\today}

\maketitle
\section{Github上のリポジトリのURL}
\url{https://github.com/BIG-POG/webpro_06}
\section{開発者向け仕様書}
\subsection{にじさんじSEEd'sメンバー一覧}
\subsubsection{概要}
本システムではVTuder事務所であるにじさんじの元SEEd's一期生の情報を管理し,
REST APIの原則に従って,データの一覧,詳細表示,追加,変更,削除の5つの操作を提供するシステムである.
\subsubsection{データ構造}
データ構造は以下の表\ref{1-1}のようになる.
\begin{table}[H]
\centering
\caption{にじさんじSEEd'sメンバーのデータ構造}
\begin{tabular}{|c|c|c|}
\hline
項目 & データ型 & 説明 \\\hline
id & 整数 & メンバーのID \\\hline
name & 文字列 & メンバーの名前 \\\hline
year & 文字列 & メンバーの活動期間 \\\hline
species & 文字列 & メンバーの種族 \\\hline
birthday & 文字列 & メンバーの誕生日 \\\hline
fanmark & 文字列 & メンバーのファンマーク \\\hline
fanname & 文字列 & メンバーのファンネーム \\\hline
\end{tabular}
\label{1-1}
\end{table}
表\ref{1-1}に示すデータは\texttt{nijisanji\_seeds}に保存し,メンバーのID,名前,活動期間,種族,誕生日,ファンマーク,ファンネームを管理する.
idは各メンバーを一意に識別するためのものであり,データの追加を行った場合には自動で設定される.

\subsubsection{ページ遷移}
本システムのページ遷移は以下の図\ref{fig:1-1}のようになる.\par
\begin{figure}[H]
\centering
\includegraphics[width=8cm]{fig/nijisanji.png}
\caption{にじさんじSEEd'sメンバーページ遷移図}
\label{fig:1-1}
\end{figure}
図\ref{fig:1-1}に示すように,トップページからにじさんじSEEd'sメンバー一覧ページへ遷移し,そこから各メンバーの詳細ページへ遷移できる.
詳細ページからは編集ページへ遷移でき,編集ページでデータを更新した後,一覧ページへ戻ることができる.また,詳細ページからメンバーの削除も可能であり,削除後は一覧ページへ戻る.
そして,メンバー一覧ページから新規メンバー登録ページへ遷移できる.\par
\subsubsection{リソースの説明}
本システムで使用する各ページのリソースのHTTPメソッドとリソース名は以下の表\ref{1-2}に示す.
\begin{table}[H]
\centering
\caption{にじさんじSEEd'sメンバーリソース一覧}
\begin{tabular}{|c|c|}
\hline
HTTPメソッド & リソース名 \\\hline
\texttt{get} & \texttt{/nijisanji} \\\hline
\texttt{get} & \texttt{/nijisanji/create}\\\hline
\texttt{get}& \texttt{/nijisanji/:number} \\\hline
\texttt{get}& \texttt{/nijisanji/delete/:number} \\\hline
\texttt{post}& \texttt{/nijisanji\_new} \\\hline
\texttt{get}& \texttt{/nijisanji/edit/:number} \\\hline
\texttt{post}& \texttt{/nijisanji/update/:number} \\
\hline
\end{tabular}
\caption{にじさんじSEEd'sメンバーリソース一覧}
\label{1-2}
\end{table}
表\ref{1-2}に示すように,各リソースはHTTPメソッドとリソース名で構成されている.
\texttt{get}メソッドはデータの取得に使用され,\texttt{post}メソッドはデータの作成や更新に使用される.
リソース名は各ページのURLパスを表している.
\texttt{/nijisanji}はメンバーの一覧を表示するページであり,サーバーから\texttt{nijisanji\_seeds}データを取得し,\texttt{nijisanji.ejs}テンプレートに渡して表示する.\par
\texttt{/nijisanji/create}は新規メンバー登録ページであり,登録用のフォームである\texttt{nijisanji\_create.html}を表示し,フォームから入力されたデータを\texttt{post}メソッドで\texttt{/nijisanji\_new}リソースに送信し,新しいメンバーを追加する.\par
\texttt{/nijisanji/:number}は各メンバーの詳細ページであり,指定されたメンバーIDに対応するデータを取得し,詳細表示用のテンプレートである\texttt{nijisanji\_detail.ejs}を表示する.\par
\texttt{/nijisanji/delete/:number}は各メンバーの削除処理を行うリソースであり,指定されたメンバーIDに対応するデータを削除し,一覧ページに遷移する.\par
\texttt{/nijisanji\_new}は新規メンバーの登録処理を行うリソースであり,登録ページから送信されたデータを受け取り,新しいメンバーを\texttt{nijisanji\_seeds}に追加し,一覧ページに遷移する.\par
\texttt{/nijisanji/edit/:number}は各メンバーの編集ページであり,指定されたメンバーIDに対応するデータを取得し,編集用のテンプレートである\texttt{nijisanji\_edit.ejs}を表示する.\par
\texttt{/nijisanji/update/:number}は各メンバーの更新処理を行うリソースであり,編集ページから送信されたデータを受け取り,対応するメンバーのデータを更新し,一覧ページに遷移する.\par








\subsection{歴代ミスターオリンピアチャンピオン一覧}
\subsubsection{概要}
本システムでは世界最高峰のボディビル大会であるミスターオリンピアの歴代チャンピオンの情報を管理し,
REST APIの原則に従って,データの一覧,詳細表示,追加,変更,削除の5つの操作を提供するシステムである.
\subsubsection{データ構造}
データ構造は以下の表\ref{2-1}のようになる.
\begin{table}[H]
\centering
\caption{歴代ミスターオリンピアチャンピオンのデータ構造}
\begin{tabular}{|c|c|c|}
\hline
項目 & データ型 & 説明 \\
\hline
id & 整数 & チャンピオンのID \\
year & 文字列 & チャンピオンの獲得年 \\
name & 文字列 & チャンピオンの名前 \\
from & 文字列 & チャンピオンの出身地 \\
height & 文字列 & チャンピオンの身長 \\
strengths & 文字列 & チャンピオンの長所 \\
\hline
\end{tabular}
\label{2-1}
\end{table}
表\ref{2-1}に示すデータは\texttt{OlympiaChampion}に保存し,チャンピオンのID,獲得年,名前,出身地,身長,長所を管理する.
idは各チャンピオンを一意に識別するためのものであり,データの追加を行った場合には自動で設定される.

\subsubsection{ページ遷移}
本システムのページ遷移は以下の図\ref{fig:2-1}のようになる.\par
\begin{figure}[H]
\centering
\includegraphics[width=8cm]{fig/Olympia.png}
\caption{歴代ミスターオリンピアチャンピオンページ遷移図}
\label{fig:2-1}
\end{figure}
図\ref{fig:2-1}に示すように,トップページから歴代チャンピオン一覧ページへ遷移し,そこから各チャンピオンの詳細ページへ遷移できる.
詳細ページからは編集ページへ遷移でき,編集ページでデータを更新した後,一覧ページへ戻ることができる.また,詳細ページからチャンピオンの削除も可能であり,削除後は一覧ページへ戻る.
そして,歴代チャンピオン一覧ページから新規のチャンピオン登録ページへ遷移できる.\par
\subsubsection{リソースの説明}
本システムで使用する各ページのリソースのHTTPメソッドとリソース名は以下の表\ref{2-2}に示す.
\begin{table}[H]
\centering
\caption{歴代ミスターオリンピアチャンピオンリソース一覧}
\begin{tabular}{|c|c|}
\hline
HTTPメソッド & リソース名 \\\hline
\texttt{get} & \texttt{/Olympia} \\\hline
\texttt{get} & \texttt{/Olympia/create}\\\hline
\texttt{get}& \texttt{/Olympia/:number} \\\hline
\texttt{get}& \texttt{/Olympia/delete/:number} \\\hline
\texttt{post}& \texttt{/Olympia\_new} \\\hline
\texttt{get}& \texttt{/Olympia/edit/:number} \\\hline
\texttt{post}& \texttt{/Olympia/update/:number} \\
\hline
\end{tabular}
\label{2-2}
\end{table}
表\ref{2-2}に示すように,各リソースはHTTPメソッドとリソース名で構成されている.
\texttt{get}メソッドはデータの取得に使用され,\texttt{post}メソッドはデータの作成や更新に使用される.
リソース名は各ページのURLパスを表している.
\texttt{/Olympia}は歴代チャンピオンの一覧を表示するページであり,サーバーから\texttt{OlympiaChampion}データを取得し,\texttt{Olympia.ejs}テンプレートに渡して表示する.
\texttt{/Olympia/create}は新規のチャンピオン登録ページであり,登録用のフォームである\texttt{Olympia\_create.html}を表示し,フォームから入力されたデータを\texttt{post}メソッドで\texttt{/Olympia\_new}リソースに送信し,新しいチャンピオンを追加する.
\texttt{/Olympia/:number}は各チャンピオンの詳細ページであり,指定されたIDに対応するデータを取得し,詳細表示用のテンプレートである\texttt{Olympia\_detail.ejs}を表示する.
\texttt{/Olympia/delete/:number}は各チャンピオンの削除処理を行うリソースであり,指定されたIDに対応するデータを削除し,一覧ページに遷移する.
\texttt{/Olympia\_new}は新規のチャンピオンの登録処理を行うリソースであり,登録ページから送信されたデータを受け取り,新しいチャンピオンを\texttt{OlympiaChampion}に追加し,一覧ページに遷移する.
\texttt{/Olympia/edit/:number}は各チャンピオンの編集ページであり,指定されたIDに対応するデータを取得し,編集用のテンプレートである\texttt{Olympia\_edit.ejs}を表示する.
\texttt{/Olympia/update/:number}は各チャンピオンの更新処理を行うリソースであり,編集ページから送信されたデータを受け取り,対応するチャンピオンのデータを更新し,一覧ページに遷移する.


\subsection{ARMORED CORE6登場AC一覧}
\subsubsection{概要}
本システムでは2023年にフロム・ソフトウェアから発売されたARMORED CORE6に登場するACの情報を管理し,
REST APIの原則に従って,データの一覧,詳細表示,追加,変更,削除の5つの操作を提供するシステムである.
\subsubsection{データ構造}
データ構造は以下の表\ref{3-1}のようになる.
\begin{table}[H]
\centering
\begin{tabular}{|c|c|c|}
\hline
項目 & データ型 & 説明 \\
\hline
id & 整数 & ACのID \\
ac\_name & 文字列 & ACの名前 \\
name & 文字列 & 搭乗者 \\
rank & 文字列 & ランク \\
affiliate & 文字列 & 所属 \\
\hline
\end{tabular}
\caption{ARMORED CORE6登場ACのデータ構造}
\label{3-1}
\end{table}
表\ref{3-1}に示すデータはACに保存し,ACのID,名前,搭乗者,ランク,所属を管理する.
idは各ACを一意に識別するためのものであり,データの追加を行った場合には自動で設定される.

\subsubsection{ページ遷移}
本システムのページ遷移は以下の図\ref{fig:3-1}のようになる.\par
\begin{figure}[H]
\centering
\includegraphics[width=8cm]{fig/Olympia.png}
\caption{ARMORED CORE6登場ACページ遷移図}
\label{fig:3-1}
\end{figure}
図\ref{fig:3-1}に示すように,トップページから登場AC一覧ページへ遷移し,そこから各ACの詳細ページへ遷移できる.
詳細ページからは編集ページへ遷移でき,編集ページでデータを更新した後,一覧ページへ戻ることができる.また,詳細ページからACの削除も可能であり,削除後は一覧ページへ戻る.
そして,登場AC一覧ページから新規のAC登録ページへ遷移できる.\par
\subsubsection{リソースの説明}
本システムで使用する各ページのリソースのHTTPメソッドとリソース名は以下の表\ref{3-2}に示す.
\begin{table}[H]
\centering
\caption{ARMORED CORE6登場ACリソース一覧}
\begin{tabular}{|c|c|}
\hline
HTTPメソッド & リソース名 \\\hline
\texttt{get} & \texttt{/ac6} \\\hline
\texttt{get} & \texttt{/ac6}\\\hline
\texttt{get}& \texttt{/ac6/:number} \\\hline
\texttt{get}& \texttt{/ac6/delete/:number} \\\hline
\texttt{post}& \texttt{/ac6\_new} \\\hline
\texttt{get}& \texttt{/ac6/edit/:number} \\\hline
\texttt{post}& \texttt{/ac6/update/:number} \\
\hline
\end{tabular}
\label{3-2}
\end{table}
表\ref{3-2}に示すように,各リソースはHTTPメソッドとリソース名で構成されている.
\texttt{get}メソッドはデータの取得に使用され,\texttt{post}メソッドはデータの作成や更新に使用される.
リソース名は各ページのURLパスを表している.
\texttt{/ac6}はACの一覧を表示するページであり,サーバーから\texttt{AC}データを取得し,\texttt{ac6.ejs}テンプレートに渡して表示する.
\texttt{/ac6/create}は新規のAC登録ページであり,登録用のフォームである\texttt{ac6\_create.html}を表示し,フォームから入力されたデータを\texttt{post}メソッドで\texttt{/ac6\_new}リソースに送信し,新しいACを追加する.
\texttt{/ac6/:number}は各ACの詳細ページであり,指定されたIDに対応するデータを取得し,詳細表示用のテンプレートである\texttt{Olympia\_detail.ejs}を表示する.
\texttt{/ac6/delete/:number}は各ACの削除処理を行うリソースであり,指定されたIDに対応するデータを削除し,一覧ページに遷移する.
\texttt{/ac6\_new}は新規のACの登録処理を行うリソースであり,登録ページから送信されたデータを受け取り,新しいACを\texttt{AC}に追加し,一覧ページに遷移する.
\texttt{/ac6/edit/:number}は各ACの編集ページであり,指定されたIDに対応するデータを取得し,編集用のテンプレートである\texttt{ac6\_edit.ejs}を表示する.
\texttt{/ac6/update/:number}は各ACの更新処理を行うリソースであり,編集ページから送信されたデータを受け取り,対応するACのデータを更新し,一覧ページに遷移する.

\newpage
\section{管理者向け仕様書}
\subsection{インストール方法}
本システムを運用するために必要なソフトウェアとそのインストール手順について説明する.
まず初めに,MacOSにパッケージ管理システムである\texttt{Homebrew}をインストールする.
ターミナルで以下のコマンドを実行する.
\begin{lstlisting}
    /bin/bash -c "$(curl -fsSL https://raw.githubusercontent.com/Homebrew/install/HEAD/install.sh)"
\end{lstlisting}
パスワードの入力を求められた場合は,MacOSのログインパスワードを入力しプロンプトが表示されるまで待つ.
次に以下の2つのコマンドを順に実行する.
\begin{lstlisting}
    ( echo; echo 'eval "$(/opt/homebrew/bin/brew shellenv)"') >> ~/.zprofile
    eval "$(/opt/homebrew/bin/brew shellenv)"
\end{lstlisting}
次に,JavaScriptの実行環境である\texttt{Node.js}を管理するために必要な\texttt{nodebrew}をインストールする.
ターミナルで以下の4つのコマンドを順に実行する.
\begin{lstlisting}
    brew install nodebrew
    nodebrew setup
    echo 'export PATH=$HOME/.nodebrew/current/bin:$PATH' >> ~/.zshrc
    source ~/.zshrc
\end{lstlisting}
次に,\texttt{Node.js}の最新版をインストールする.
ターミナルで以下の2つのコマンドを順に実行する.
\begin{lstlisting}
    nodebrew install stable
    nodebrew ls
\end{lstlisting}
ここで表示された最新版のバージョン番号を確認し,以下のコマンドを実行してインストールした最新版を使用するように設定する.仕様書ではバージョン番号を\texttt{v24.3.0}と仮定する.
最後に以下の2つのコマンドを順に実行する.
\begin{lstlisting}  
    nodebrew use v24.3.0
    npm install -g npm
\end{lstlisting}
\subsection{起動方法}
本システムを起動する手順について説明する.
まず初めに,ターミナルを起動し,本システムのプロジェクトディレクトリに移動する.
次に,以下のコマンドを実行し本システムを起動する.
\begin{lstlisting}
    node app5.js
\end{lstlisting}
ターミナルに\texttt{Example app listening on port 8080!}と表示されたら起動できている状態である.
\subsection{起動できない場合}
本システムが起動できない場合の対処方法について説明する.
まず初めに,ターミナルで既に本システムや他のアプリケーションが使用しているポート番号が競合していないかや適したディレクトリに移動しているかを確認し再度起動を試みる.
次に,\texttt{Node.js}や\texttt{npm}のバージョンが古い場合,最新バージョンにアップデートし再度起動を試みる.
最後に,それでも起動できない場合は,github上から本システムのリポジトリを再度クローンし,手順に従ってインストールと起動を試みる.

\subsection{終了方法}
本システムを終了する手順について説明する.
ターミナルで本システムが起動している状態で,\texttt{Ctrl + C}キーを同時に押すことで終了できる.    

\subsection{分かっている不具合}
現時点で本システムにおいて確認されている不具合として情報の追加や編集などの更新処理を行った際に,
システムの再起動を行うと更新した情報が消失してしまう問題がある.
\newpage

\section{利用者向け仕様書}
\subsection{システムの概要}
本システムは,VTuder事務所であるにじさんじの元SEEd's一期生の情報を管理し,一覧表示,詳細表示,追加,変更,削除の5つの操作を提供するシステムである.

\subsection{システムの利用方法}
本システムの利用方法について説明する.
まず初めに,ウェブブラウザを起動し,アドレスバーに\texttt{http://localhost:8080/nijisanji}と入力し,にじさんじSEEd'sメンバー一覧ページにアクセスすると図\ref{fig:4-1}のような画面が表示される.
\begin{figure}[H]
\centering
\includegraphics[width=5cm]{fig/nijisanji1.png}
\caption{にじさんじSEEd'sメンバー一覧ページ}
\label{fig:4-1}
\end{figure}
図\ref{fig:4-1}に示すように,メンバーの一覧が表示され,各メンバーの名前をクリックすることで詳細ページへ遷移できる.
遷移した詳細ページでは図\ref{fig:4-2}のような画面が表示される.
\begin{figure}[H]
\centering
\includegraphics[width=5cm]{fig/nijisanji2.png}
\caption{にじさんじSEEd'sメンバー詳細ページ}
\label{fig:4-2}
\end{figure}
図\ref{fig:4-2}に示すように,メンバーの詳細情報が表示され,編集ボタンをクリックすることで編集ページへ遷移できる.
遷移した編集ページでは図\ref{fig:4-3}のような画面が表示される.
\begin{figure}[H]
\centering
\includegraphics[width=5cm]{fig/nijisanji3.png}
\caption{にじさんじSEEd'sメンバー編集ページ}
\label{fig:4-3}
\end{figure}
図\ref{fig:4-3}に示すように,メンバーの情報を編集でき,更新ボタンをクリックすることで変更内容が保存され一覧ページへ戻る.
更新した詳細ページは図\ref{fig:4-4}のように表示される.
\begin{figure}[H]
\centering
\includegraphics[width=5cm]{fig/nijisanji4.png}
\caption{にじさんじSEEd'sメンバー詳細ページ(更新後)}
\label{fig:4-4}
\end{figure}
図\ref{fig:4-4}に示すように,更新した内容が反映されている.
また,削除ボタンをクリックすることでメンバーの情報を削除でき,一覧ページへ戻る.
さらに,メンバー一覧ページの新規登録ボタンをクリックすることで新規メンバー登録ページへ遷移できる.
遷移した新規メンバー登録ページでは図\ref{fig:4-5}のような画面が表示される.
\begin{figure}[H]
\centering
\includegraphics[width=5cm]{fig/nijisanji5.png}
\caption{にじさんじSEEd'sメンバー新規登録ページ}
\label{fig:4-5}
\end{figure}
図\ref{fig:4-5}に示すように,新規メンバーの情報を入力でき,登録ボタンをクリックすることで新しいメンバーが追加され一覧ページへ戻る.
追加したメンバーの一覧ページは図\ref{fig:4-6}のように表示される.
\begin{figure}[H]
\centering
\includegraphics[width=5cm]{fig/nijisanji6.png}
\caption{にじさんじSEEd'sメンバー一覧ページ(追加後)}
\label{fig:4-6}
\end{figure}
図\ref{fig:4-6}に示すように,追加したメンバーが一覧に表示されている.
\end{document}