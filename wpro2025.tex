\documentclass[uplatex]{jsarticle}
\usepackage{amsmath}
\usepackage{graphicx}
\usepackage[dvipdfmx]{color}

\usepackage[uplatex,deluxe]{otf} % UTF
\usepackage[noalphabet]{pxchfon} % must be after otf package
\setcounter{tocdepth}{3}
\usepackage{float}
\usepackage{moreverb}
\usepackage{lscape}
%\pagestyle{empty}
%\usepackage{wrapfig}

%\usepackage{EasyLayout}
\usepackage{listings}
\usepackage{ascmac}

% --- 追加設定 ---
\usepackage{geometry}
\usepackage{xcolor}
\usepackage[hidelinks]{hyperref}
\usepackage{verbatimbox}
\usepackage{color}

% --- ページレイアウト ---
\geometry{a4paper, margin=1in}

% --- listings (コードブロック) の設定 ---
\lstset{
    language=C,
    basicstyle=\small\ttfamily,
    keywordstyle=\color{blue},
    commentstyle=\color{green!40!black},
    stringstyle=\color{purple},
    showstringspaces=false,
    breaklines=true,
    frame=tb,
    backgroundcolor=\color{black!5},
    captionpos=b,
    numberstyle=\tiny\color{gray},
    numbers=left,
    stepnumber=1
}

% --- ドキュメント開始 ---
\begin{document}
\title{webプログラミング仕様書}
\author{25G1088 津村 大樹}
\date{\today}

\maketitle
\section{Github上のリポジトリのURL}
\url{https://github.com/BIG-POG/webpro_06}
\section{開発者向け仕様書}
\subsection{にじさんじSEEd'sメンバー一覧}
\subsubsection{概要}
本システムではVTuder事務所であるにじさんじの元SEEd's一期生の情報を管理し,
REST APIの原則に従って,データの一覧,詳細表示,追加,変更,削除の5つの操作を提供するシステムである.
\subsubsection{データ構造}
データ構造は以下の表\ref{1-1}のようになる.
\begin{table}[H]
\centering
\caption{にじさんじSEEd'sメンバーのデータ構造}
\begin{tabular}{|c|c|c|}
\hline
項目 & データ型 & 説明 \\\hline
id & 整数 & メンバーのID \\\hline
name & 文字列 & メンバーの名前 \\\hline
year & 文字列 & メンバーの活動期間 \\\hline
species & 文字列 & メンバーの種族 \\\hline
birthday & 文字列 & メンバーの誕生日 \\\hline
fanmark & 文字列 & メンバーのファンマーク \\\hline
fanname & 文字列 & メンバーのファンネーム \\\hline
\end{tabular}
\label{1-1}
\end{table}
表\ref{1-1}に示すデータは\texttt{nijisanji\_seeds}に保存し,メンバーのID,名前,活動期間,種族,誕生日,ファンマーク,ファンネームを管理する.
idは各メンバーを一意に識別するためのものであり,データの追加を行った場合には自動で設定される.

\subsubsection{ページ遷移}
本システムのページ遷移は以下の図\ref{fig:1-1}のようになる.\par
\begin{figure}[H]
\centering
\includegraphics[width=8cm]{fig/nijisanji.png}
\caption{にじさんじSEEd'sメンバーページ遷移図}
\label{fig:1-1}
\end{figure}
図\ref{fig:1-1}に示すように,トップページからにじさんじSEEd'sメンバー一覧ページへ遷移し,そこから各メンバーの詳細ページへ遷移できる.
詳細ページからは編集ページへ遷移でき,編集ページでデータを更新した後,一覧ページへ戻ることができる.また,詳細ページからメンバーの削除も可能であり,削除後は一覧ページへ戻る.
そして,メンバー一覧ページから新規メンバー登録ページへ遷移できる.\par
\subsubsection{リソースの説明}
本システムで使用する各ページのリソースのHTTPメソッドとリソース名は以下の表\ref{1-2}に示す.
\begin{table}[H]
\centering
\caption{にじさんじSEEd'sメンバーリソース一覧}
\begin{tabular}{|c|c|}
\hline
HTTPメソッド & リソース名 \\\hline
\texttt{get} & \texttt{/nijisanji} \\\hline
\texttt{get} & \texttt{/nijisanji/create}\\\hline
\texttt{get}& \texttt{/nijisanji/:number} \\\hline
\texttt{get}& \texttt{/nijisanji/delete/:number} \\\hline
\texttt{post}& \texttt{/nijisanji\_new} \\\hline
\texttt{get}& \texttt{/nijisanji/edit/:number} \\\hline
\texttt{post}& \texttt{/nijisanji/update/:number} \\
\hline
\end{tabular}
\caption{にじさんじSEEd'sメンバーリソース一覧}
\label{1-2}
\end{table}
表\ref{1-2}に示すように,各リソースはHTTPメソッドとリソース名で構成されている.
\texttt{get}メソッドはデータの取得に使用され,\texttt{post}メソッドはデータの作成や更新に使用される.
リソース名は各ページのURLパスを表している.
\texttt{/nijisanji}はメンバーの一覧を表示するページであり,サーバーから\texttt{nijisanji\_seeds}データを取得し,\texttt{nijisanji.ejs}テンプレートに渡して表示する.\par
\texttt{/nijisanji/create}は新規メンバー登録ページであり,登録用のフォームである\texttt{nijisanji\_create.html}を表示し,フォームから入力されたデータを\texttt{post}メソッドで\texttt{/nijisanji}リソースに送信し,新しいメンバーを追加する.\par
\texttt{/nijisanji/:number}は各メンバーの詳細ページであり,指定されたメンバーIDに対応するデータを取得し,詳細表示用のテンプレートである\texttt{nijisanji\_detail.ejs}を表示する.\par
\texttt{/nijisanji/delete/:number}は各メンバーの削除処理を行うリソースであり,指定されたメンバーIDに対応するデータを削除し,一覧ページに遷移する.\par
\texttt{/nijisanji\_new}は新規メンバーの登録処理を行うリソースであり,登録ページから送信されたデータを受け取り,新しいメンバーを\texttt{nijisanji\_seeds}に追加し,一覧ページに遷移する.\par
\texttt{/nijisanji/edit/:number}は各メンバーの編集ページであり,指定されたメンバーIDに対応するデータを取得し,編集用のテンプレートである\texttt{nijisanji\_edit.ejs}を表示する.\par
\texttt{/nijisanji/update/:number}は各メンバーの更新処理を行うリソースであり,編集ページから送信されたデータを受け取り,対応するメンバーのデータを更新し,一覧ページに遷移する.\par








\subsection{歴代ミスターオリンピアチャンピオン一覧}
\subsubsection{概要}
本システムでは世界最高峰のボディビル大会であるミスターオリンピアの歴代チャンピオンの情報を管理し,
REST APIの原則に従って,データの一覧,詳細表示,追加,変更,削除の5つの操作を提供するシステムである.
\subsubsection{データ構造}
データ構造は以下の表\ref{2-1}のようになる.
\begin{table}[H]
\centering
\caption{歴代ミスターオリンピアチャンピオンのデータ構造}
\begin{tabular}{|c|c|c|}
\hline
項目 & データ型 & 説明 \\
\hline
id & 整数 & チャンピオンのID \\
year & 文字列 & チャンピオンの獲得年 \\
name & 文字列 & チャンピオンの名前 \\
from & 文字列 & チャンピオンの出身地 \\
height & 文字列 & チャンピオンの身長 \\
strengths & 文字列 & チャンピオンの長所 \\
\hline
\end{tabular}
\label{2-1}
\end{table}
表\ref{2-1}に示すデータは\texttt{OlympiaChampion}に保存し,チャンピオンのID,獲得年,名前,出身地,身長,長所を管理する.
idは各チャンピオンを一意に識別するためのものであり,データの追加を行った場合には自動で設定される.

\subsubsection{ページ遷移}
本システムのページ遷移は以下の図\ref{fig:2-1}のようになる.\par
\begin{figure}[H]
\centering
\includegraphics[width=8cm]{fig/Olympia.png}
\caption{歴代ミスターオリンピアチャンピオンページ遷移図}
\label{fig:2-1}
\end{figure}
図\ref{fig:2-1}に示すように,トップページから歴代チャンピオン一覧ページへ遷移し,そこから各チャンピオンの詳細ページへ遷移できる.
詳細ページからは編集ページへ遷移でき,編集ページでデータを更新した後,一覧ページへ戻ることができる.また,詳細ページからチャンピオンの削除も可能であり,削除後は一覧ページへ戻る.
そして,歴代チャンピオン一覧ページから新規のチャンピオン登録ページへ遷移できる.\par
\subsubsection{リソースの説明}
本システムで使用する各ページのリソースのHTTPメソッドとリソース名は以下の表\ref{1-2}に示す.
\begin{table}[H]
\centering
\caption{歴代ミスターオリンピアチャンピオンリソース一覧}
\begin{tabular}{|c|c|}
\hline
HTTPメソッド & リソース名 \\\hline
\texttt{get} & \texttt{/Olympia} \\\hline
\texttt{get} & \texttt{/Olympia/create}\\\hline
\texttt{get}& \texttt{/Olympia/:number} \\\hline
\texttt{get}& \texttt{/Olympia/delete/:number} \\\hline
\texttt{post}& \texttt{/Olympia\_new} \\\hline
\texttt{get}& \texttt{/Olympia/edit/:number} \\\hline
\texttt{post}& \texttt{/Olympia/update/:number} \\
\hline
\end{tabular}
\label{2-2}
\end{table}
表\ref{1-2}に示すように,各リソースはHTTPメソッドとリソース名で構成されている.
\texttt{get}メソッドはデータの取得に使用され,\texttt{post}メソッドはデータの作成や更新に使用される.
リソース名は各ページのURLパスを表している.
\texttt{/Olympia}は歴代チャンピオンの一覧を表示するページであり,サーバーから\texttt{OlympiaChampion}データを取得し,\texttt{Olympia.ejs}テンプレートに渡して表示する.
\texttt{/Olympia/create}は新規のチャンピオン登録ページであり,登録用のフォームである\texttt{Olympia\_create.html}を表示し,フォームから入力されたデータを\texttt{post}メソッドで\texttt{/Olympia}リソースに送信し,新しいチャンピオンを追加する.
\texttt{/Olympia/:number}は各チャンピオンの詳細ページであり,指定されたIDに対応するデータを取得し,詳細表示用のテンプレートである\texttt{Olympia\_detail.ejs}を表示する.
\texttt{/Olympia/delete/:number}は各チャンピオンの削除処理を行うリソースであり,指定されたIDに対応するデータを削除し,一覧ページに遷移する.
\texttt{/Olympia\_new}は新規のチャンピオンの登録処理を行うリソースであり,登録ページから送信されたデータを受け取り,新しいチャンピオンを\texttt{OlympiaChampion}に追加し,一覧ページに遷移する.
\texttt{/Olympia/edit/:number}は各チャンピオンの編集ページであり,指定されたIDに対応するデータを取得し,編集用のテンプレートである\texttt{Olympia\_edit.ejs}を表示する.
\texttt{/Olympia/update/:number}は各チャンピオンの更新処理を行うリソースであり,編集ページから送信されたデータを受け取り,対応するチャンピオンのデータを更新し,一覧ページに遷移する.


\subsection{ARMORED CORE6登場AC一覧}
\subsubsection{概要}
本システムでは2023年にフロム・ソフトウェアから発売されたARMORED CORE6に登場するACの情報を管理し,
REST APIの原則に従って,データの一覧,詳細表示,追加,変更,削除の5つの操作を提供するシステムである.
\subsubsection{データ構造}
データ構造は以下の表\ref{3-1}のようになる.
\begin{table}[H]
\centering
\begin{tabular}{|c|c|c|}
\hline
項目 & データ型 & 説明 \\
\hline
id & 整数 & ACのID \\
ac\_name & 文字列 & ACの名前 \\
name & 文字列 & 搭乗者 \\
rank & 文字列 & ランク \\
affiliate & 文字列 & 所属 \\
\hline
\end{tabular}
\caption{ARMORED CORE6登場ACのデータ構造}
\label{3-1}
\end{table}
表\ref{3-1}に示すデータはACに保存し,ACのID,名前,搭乗者,ランク,所属を管理する.
idは各ACを一意に識別するためのものであり,データの追加を行った場合には自動で設定される.

\subsubsection{ページ遷移}
本システムのページ遷移は以下の図\ref{fig:2-1}のようになる.\par
\begin{figure}[H]
\centering
\includegraphics[width=8cm]{fig/Olympia.png}
\caption{歴代ミスターオリンピアチャンピオンページ遷移図}
\label{fig:3-1}
\end{figure}
図\ref{fig:2-1}に示すように,トップページから歴代チャンピオン一覧ページへ遷移し,そこから各チャンピオンの詳細ページへ遷移できる.
詳細ページからは編集ページへ遷移でき,編集ページでデータを更新した後,一覧ページへ戻ることができる.また,詳細ページからチャンピオンの削除も可能であり,削除後は一覧ページへ戻る.
そして,歴代チャンピオン一覧ページから新規のチャンピオン登録ページへ遷移できる.\par
\subsubsection{リソースの説明}
本システムで使用する各ページのリソースのHTTPメソッドとリソース名は以下の表\ref{1-2}に示す.
\begin{table}[H]
\centering
\caption{歴代ミスターオリンピアチャンピオンリソース一覧}
\begin{tabular}{|c|c|}
\hline
HTTPメソッド & リソース名 \\\hline
\texttt{get} & \texttt{/Olympia} \\\hline
\texttt{get} & \texttt{/Olympia/create}\\\hline
\texttt{get}& \texttt{/Olympia/:number} \\\hline
\texttt{get}& \texttt{/Olympia/delete/:number} \\\hline
\texttt{post}& \texttt{/Olympia\_new} \\\hline
\texttt{get}& \texttt{/Olympia/edit/:number} \\\hline
\texttt{post}& \texttt{/Olympia/update/:number} \\
\hline
\end{tabular}
\label{3-2}
\end{table}
表\ref{1-2}に示すように,各リソースはHTTPメソッドとリソース名で構成されている.
\texttt{get}メソッドはデータの取得に使用され,\texttt{post}メソッドはデータの作成や更新に使用される.
リソース名は各ページのURLパスを表している.
\texttt{/Olympia}は歴代チャンピオンの一覧を表示するページであり,サーバーから\texttt{OlympiaChampion}データを取得し,\texttt{Olympia.ejs}テンプレートに渡して表示する.
\texttt{/Olympia/create}は新規のチャンピオン登録ページであり,登録用のフォームである\texttt{Olympia\_create.html}を表示し,フォームから入力されたデータを\texttt{post}メソッドで\texttt{/Olympia}リソースに送信し,新しいチャンピオンを追加する.
\texttt{/Olympia/:number}は各チャンピオンの詳細ページであり,指定されたIDに対応するデータを取得し,詳細表示用のテンプレートである\texttt{Olympia\_detail.ejs}を表示する.
\texttt{/Olympia/delete/:number}は各チャンピオンの削除処理を行うリソースであり,指定されたIDに対応するデータを削除し,一覧ページに遷移する.
\texttt{/Olympia\_new}は新規のチャンピオンの登録処理を行うリソースであり,登録ページから送信されたデータを受け取り,新しいチャンピオンを\texttt{OlympiaChampion}に追加し,一覧ページに遷移する.
\texttt{/Olympia/edit/:number}は各チャンピオンの編集ページであり,指定されたIDに対応するデータを取得し,編集用のテンプレートである\texttt{Olympia\_edit.ejs}を表示する.
\texttt{/Olympia/update/:number}は各チャンピオンの更新処理を行うリソースであり,編集ページから送信されたデータを受け取り,対応するチャンピオンのデータを更新し,一覧ページに遷移する.
\end{document}